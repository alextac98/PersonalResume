\RequirePackage{ifxetex,ifluatex}
\newif\ifxetexorluatex
\ifxetex
  \xetexorluatextrue
\else
  \ifluatex
    \xetexorluatextrue
  \else
    \xetexorluatexfalse
  \fi
\fi

% Checks to see if correct compiler, then enables academicons
%% See texdoc.net/pkg/fontawecome and http://texdoc.net/pkg/academicons for full list of symbols. You MUST compile with XeLaTeX or LuaLaTeX if you want to use academicons.
\ifxetexorluatex
    \documentclass[12pt,letterpaper,ragged2e, academicons]{Resume}
    % "normalphoto" => Normal photo, not cropped
    % "ragged2e"    => Cropped photo to circle
\else
    \documentclass[12pt, letterpaper, ragged2e]{Resume}
\fi

% Change the page layout if you need to
% \geometry{left=1cm,right=9cm,marginparwidth=6.8cm,marginparsep=1.2cm,top=1.25cm,bottom=1.25cm}

% Change the font if you want to, depending on whether
% you're using pdflatex or xelatex/lualatex
\ifxetexorluatex
  % If using xelatex or lualatex:
  \setmainfont{Carlito}
\else
  % If using pdflatex:
  \usepackage[utf8]{inputenc}
  \usepackage[T1]{fontenc}
  \usepackage[default]{lato}
\fi

\usepackage{multicol}
\usepackage{parallel}
\usepackage[skins,breakable]{tcolorbox}
\usepackage{paracol}
\usepackage{parcolumns}

\raggedbottom

% Set the desired heading colors
\definecolor{headingColor}{HTML}{000000}
\definecolor{accentColor}{HTML}{ff0000}
\definecolor{emphasisColor}{HTML}{2E2E2E}
\definecolor{bodyColor}{HTML}{666666}

\colorlet{heading}{headingColor}
\colorlet{accent}{accentColor}
\colorlet{emphasis}{emphasisColor}
\colorlet{body}{bodyColor}

% Change the bullets for itemize and rating marker
% for \cvskill if you want to
\renewcommand{\itemmarker}{{\small\textbullet}}
\renewcommand{\ratingmarker}{\faCircle}

%%% sample.bib contains your publications
%\addbibresource{Publications.bib}

\begin{document}
    \name{Alex Tacescu}
    \phone{(559) 301-6222}
    \email{alextac98@gmail.com}
    \website{www.alextac.com} 
    %
    % Create Heading
    \makeheading \\[-0.6\baselineskip]
    %    
    % Start body
    \newline
    \cvsection{Qualifications Summary}
Robotics engineering student with significant experience in 3D CAD design and extensive knowledge of
multiple programing languages in multidisciplinary applications
    \bigskip\newline
    \resumesection{Technical Skills}{}
\skillsection{Robotics}{Software Development, Mechanical Design, Electrical Design, Agile Project Management (SCRUM)}

\skillsection{3D CAD}{Design and Simulation in Autodesk Inventor [7 years], Dassault SolidWorks [5 years], and PTC Creo Parametric (ProE) [1 year]}

\skillsection{Programming}{ROS (C++ \& Python), Android (Java), Pipelines (in Unix/ Linux) and Git}

\skillsection{Microprocessor \& Single-Board Computers}{Raspberry Pi, BeagleBone Black/Blue, NVIDIA TX1/2,ESP32 \& ESP8266, Device Trees in Linux}

\skillsection{Other Software Experience}{Linux (Debian/Ubuntu), MathCAD and MATLAB, Adobe Creative Suite(Photoshop \& Premiere), Excel Macros Programming}
\\
    \bigskip\newline
    \resumesection{Education}{}
\event{M.S. in Computer Science}{Stanford University}{Sept 1997 -- June 1999}{}
\divider

\event{B.S. in Symbolic Systems}{Stanford University}{Sept 1993 -- June 1997}{}
    \bigskip\newline
    \resumesection{Work Experience}{References Available upon Request}
\job{Advanced Robotics R\&D Internship}{Amazon Robotics}{May 2020 - August 2020}{Boston, MA}{Sections/Logos/AR_Logo.jpg}
    \begin{itemize}
        \item Analyzed robot arm workcell faults and collected data to SQL database to display it to dashboards using Apache Superset. Presented patterns and solutions and collaborated with leadership team in decision making.
        \item Designed and deployed workcell fault diagnosis and mitigation systems, reducing downtime by 60\% (initial tests).
        \item Developed vision system + path planning system for FANUC robots to recover from workcell faults by picking up dropped products.
    \end{itemize}
\divider
\job{Advanced Robotics R\&D Co-Op}{Amazon Robotics}{Aug 2019 - Dec 2019}{Boston, MA}{Sections/Logos/AR_Logo.jpg}
    \begin{itemize}
        \item Automated 3D Packing using a vision-guided Universal Robotics arm in Python and C++.
        \item Developed perception algorithms using 3D cameras and the PointCloudLibrary
    \end{itemize}
\divider \\
\job{Powertrain Integration Engineering Internship}{Tesla}{Summer 2019}{Palo Alto, CA}{Sections/Logos/Tesla_Logo.jpg}
    \begin{itemize}
        \item Designed and deployed new automated platform that reports reoccurring issues using anonymous fleet analytics for over 600,000 cars
        \item System was utilized by Tesla powertrain software developers to identify and debug several user-facing issues
        \item Developed in Python in Jupyter Lab using PySpark and SQL to pull and analyze fleet data. Created user interface and dashboards using Apache Superset, automated by Apache Airflow
    \end{itemize}
\divider \\
\job{Powertrain Integration Engineering Internship}
    {Tesla}
    {Summer 2018}
    {Palo Alto, CA}
    {Sections/Logos/Tesla_Logo.jpg}
    \begin{itemize}
        \item Responsible for troubleshooting battery thermal system issues and developing test stands for Model S/X, Model 3, Semi-Truck, and other products
        \item Developed software components for testing, collecting data over CAN networks
        \item Discovered and fixed 3 issues in critical systems such as the battery and powertrain thermal system and the high voltage system
        \item Identified a problem and implemented a change in 2 assembly cells that increased Model 3 end-of-line production by 45\%
    \end{itemize}
\divider \\
\job{Student Assistant}
    {Worcester Polytechnic Institute}
    {Aug 2018 -- Current}
    {Worcester, MA}
    {Sections/Logos/WPI_Logo.png}
    \begin{itemize}
        \item Student Teaching Assistant for WPI’s Junior year Robotics classes (RBE 3001 \& 3002) focusing on robotic manipulation, dynamics, machine vision, path-planning, \& other advanced concepts
    \end{itemize}

% Updated 12/19/2020 for 2021
    \bigskip
    \resumesection{Projects}{}
\project{SmallKat Major Qualifying Project}{2018-Present}{is a quadrupedal robotic platform designed for research and development of multipedal robotic systems. SmallKat is 3D printed, open source, and contains fully custom electronics. I am developing the high-level software, including footstep planning, path planning, forward/inverse kinematics, machine vision, and networking systems for off-platform debugging. To learn more, please visit \\ \href{https://www.alextac.com/smallkat}{www.alextac.com/smallkat}}
\project{Project Maverick}{2015-Present}{is an award-winning omni-directional robotic system that provides mobility for people with walking disabilities. The drive system allows the user to move in any direction using 4 steering and 4 driving electronically synchronized motors, creating the same degrees of motion as an able person. It was	designed, built, and programmed as a personal project, initially with Java and then converted to ROS (C++ \& Python). To learn more, please visit \href{https://www.pmaverick.com}{www.pmaverick.com}}
\project{Poverty Stoplight Interactive Qualifying Project}{2017-2018}{is an Android application for social workers in Paraguay to better help people in poverty. The application was designed for Fundación Paraguay and Poverty Stoplight and consisted of developing a REST API and an Android application capable of syncing sensitive family data with a secure server. To learn more, please visit \href{https://www.alextac.com/stoplight-iqp}{www.alextac.com/stoplight-iqp}}
\project{NASA Space Robotics Challenge}{2016-17}{is a competition to develop software for NASA’s humanoid robot Valkyrie. Developed footstep motion planning, optimized cycle-speed, and tested in ROS, C++, and Python with Gazebo as a member of the WPI Humanoid Robotics Lab. To learn more, please visit \href{https://www.alextac.com/src}{www.alextac.com/src}}
\project{Project Drogo}{2017}{is a wearable embedded system accompanied by a smart-phone app designed to assist elderly people through post-hip surgery recovery. It combines 2 goals of post-surgery medicine: preventing prohibited motions and guiding the user through physical therapy and rehab. Developed on a team of 4 students as a part of the hackathon Health Hacks RI. To learn more, please visit \href{https://www.alextac.com/drogo}{www.alextac.com/drogo}}
\project{Project Pather}{2017}{is a kiosk mapping software developed to provide directions to Bringham and Women’s Hospital visitors. It has contextual search as well as the capability to send users directions via text message or email. It is written in Java and JavaFX, with a SQL backend, and was developed on an 8-person team for a school project. To learn more, please visit \href{https://www.alextac.com/pather}{www.alextac.com/pather}}
\project{FIRST FRC Robotics Team 2761}{2012-16}{4 cumulative seasons with the team. Designed, built, programmed, and tested 5 full-size robots. To learn more, please visit \href{https://www.alextac.com/frc}{www.alextac.com/frc}}

    \bigskip\newline\newline
    \resumesection{Awards}{For an updated list, please visit \href{https://www.alextac.com/awards}{www.alextac.com/awards}}
	\award{2019}{SmallKat Major Qualifying Project (MQP) Honerable Mention (2nd best capstone project in Robotics Engineering Department)}
	\award{2018}{Dean’s List at WPI (Spring 2018)}
	\award{2018}{Inducted in Rho Beta Epsilon Robotics Engineering Honor Society}
	\award{2017}{1st Place at HealthHacksRI at the University of Rhode Island for Project Drogo}
	\award{2017}{NASA Space Robotics Challenge Team Finalist}
	\award{2016}{2nd Place at the Intel International Science and Engineering Fair (ISEF) in the category of Applied Mechanics}
	\award{2016}{Google International Science Fair Regional Finalist (top 100 in the world)}
	\award{2016}{International Council on Systems Engineering (INCOSE) First Award for “best interdisciplinary project that can produce technologically appropriate solution that meet societal needs” at the ISEF}
	\award{2016}{GE Fallonventions Award and participation on NBC’s Tonight Show starring Jimmy Fallon (aired on April 11, 2016). \href{https://alextac.com/personal/awards/major-awards/ge-fallonventions-on-the-tonight-show-starring-jimmy-fallon/}{See it here}}
	\award{2015 \& 2016}{Sweepstakes Award winner (1st place overall) and 1st place in Engineering at the Central California Science, Math, and Engineering Fair}
	\award{2015}{National Honor Society Inductee and California Scholarship Federation Member}
	\award{2015}{Institute of Electrical and Electronics Engineers President’s Scholarship Award at Intel Science and Engineering Fair for “an outstanding project demonstrating an understanding of electrical engineering, electronics engineering, and computer science.”}
	\award{2015}{1st place in the category of Applied Mechanics at the California State Science Fair}
    \bigskip
    \resumesection{Leadership Experience}{}
	\leadership{2019-2020}{Officer board member for the Rho Beta Epsilon Robotics Honor Society}
	\newline
	\leadership{2016}{Leadership Practice at WPI: analyzed business and leadership practices for an on-campus organization with Prof. Sharon Wulf}
	\newline
	\leadership{2012-16}{ Lead Technical Director, Build and Pit Team Leader for FIRST FRC Robotics Team 2761}
    \bigskip\newline
    \resumesection{Hobbies}{}
Robots, Coding, Tennis, Ultimate Frisbee, Skiing, Fishing, Camping
    
%    \columnratio{0.60}
%	\begin{paracol}{2}
%		\begin{leftcolumn}
%			\resumesection{Work Experience}{References Available upon Request}
\job{Advanced Robotics R\&D Internship}{Amazon Robotics}{May 2020 - August 2020}{Boston, MA}{Sections/Logos/AR_Logo.jpg}
    \begin{itemize}
        \item Analyzed robot arm workcell faults and collected data to SQL database to display it to dashboards using Apache Superset. Presented patterns and solutions and collaborated with leadership team in decision making.
        \item Designed and deployed workcell fault diagnosis and mitigation systems, reducing downtime by 60\% (initial tests).
        \item Developed vision system + path planning system for FANUC robots to recover from workcell faults by picking up dropped products.
    \end{itemize}
\divider
\job{Advanced Robotics R\&D Co-Op}{Amazon Robotics}{Aug 2019 - Dec 2019}{Boston, MA}{Sections/Logos/AR_Logo.jpg}
    \begin{itemize}
        \item Automated 3D Packing using a vision-guided Universal Robotics arm in Python and C++.
        \item Developed perception algorithms using 3D cameras and the PointCloudLibrary
    \end{itemize}
\divider \\
\job{Powertrain Integration Engineering Internship}{Tesla}{Summer 2019}{Palo Alto, CA}{Sections/Logos/Tesla_Logo.jpg}
    \begin{itemize}
        \item Designed and deployed new automated platform that reports reoccurring issues using anonymous fleet analytics for over 600,000 cars
        \item System was utilized by Tesla powertrain software developers to identify and debug several user-facing issues
        \item Developed in Python in Jupyter Lab using PySpark and SQL to pull and analyze fleet data. Created user interface and dashboards using Apache Superset, automated by Apache Airflow
    \end{itemize}
\divider \\
\job{Powertrain Integration Engineering Internship}
    {Tesla}
    {Summer 2018}
    {Palo Alto, CA}
    {Sections/Logos/Tesla_Logo.jpg}
    \begin{itemize}
        \item Responsible for troubleshooting battery thermal system issues and developing test stands for Model S/X, Model 3, Semi-Truck, and other products
        \item Developed software components for testing, collecting data over CAN networks
        \item Discovered and fixed 3 issues in critical systems such as the battery and powertrain thermal system and the high voltage system
        \item Identified a problem and implemented a change in 2 assembly cells that increased Model 3 end-of-line production by 45\%
    \end{itemize}
\divider \\
\job{Student Assistant}
    {Worcester Polytechnic Institute}
    {Aug 2018 -- Current}
    {Worcester, MA}
    {Sections/Logos/WPI_Logo.png}
    \begin{itemize}
        \item Student Teaching Assistant for WPI’s Junior year Robotics classes (RBE 3001 \& 3002) focusing on robotic manipulation, dynamics, machine vision, path-planning, \& other advanced concepts
    \end{itemize}

% Updated 12/19/2020 for 2021
%				\smallskip
%			\resumesection{Projects}{}
\project{SmallKat Major Qualifying Project}{2018-Present}{is a quadrupedal robotic platform designed for research and development of multipedal robotic systems. SmallKat is 3D printed, open source, and contains fully custom electronics. I am developing the high-level software, including footstep planning, path planning, forward/inverse kinematics, machine vision, and networking systems for off-platform debugging. To learn more, please visit \\ \href{https://www.alextac.com/smallkat}{www.alextac.com/smallkat}}
\project{Project Maverick}{2015-Present}{is an award-winning omni-directional robotic system that provides mobility for people with walking disabilities. The drive system allows the user to move in any direction using 4 steering and 4 driving electronically synchronized motors, creating the same degrees of motion as an able person. It was	designed, built, and programmed as a personal project, initially with Java and then converted to ROS (C++ \& Python). To learn more, please visit \href{https://www.pmaverick.com}{www.pmaverick.com}}
\project{Poverty Stoplight Interactive Qualifying Project}{2017-2018}{is an Android application for social workers in Paraguay to better help people in poverty. The application was designed for Fundación Paraguay and Poverty Stoplight and consisted of developing a REST API and an Android application capable of syncing sensitive family data with a secure server. To learn more, please visit \href{https://www.alextac.com/stoplight-iqp}{www.alextac.com/stoplight-iqp}}
\project{NASA Space Robotics Challenge}{2016-17}{is a competition to develop software for NASA’s humanoid robot Valkyrie. Developed footstep motion planning, optimized cycle-speed, and tested in ROS, C++, and Python with Gazebo as a member of the WPI Humanoid Robotics Lab. To learn more, please visit \href{https://www.alextac.com/src}{www.alextac.com/src}}
\project{Project Drogo}{2017}{is a wearable embedded system accompanied by a smart-phone app designed to assist elderly people through post-hip surgery recovery. It combines 2 goals of post-surgery medicine: preventing prohibited motions and guiding the user through physical therapy and rehab. Developed on a team of 4 students as a part of the hackathon Health Hacks RI. To learn more, please visit \href{https://www.alextac.com/drogo}{www.alextac.com/drogo}}
\project{Project Pather}{2017}{is a kiosk mapping software developed to provide directions to Bringham and Women’s Hospital visitors. It has contextual search as well as the capability to send users directions via text message or email. It is written in Java and JavaFX, with a SQL backend, and was developed on an 8-person team for a school project. To learn more, please visit \href{https://www.alextac.com/pather}{www.alextac.com/pather}}
\project{FIRST FRC Robotics Team 2761}{2012-16}{4 cumulative seasons with the team. Designed, built, programmed, and tested 5 full-size robots. To learn more, please visit \href{https://www.alextac.com/frc}{www.alextac.com/frc}}

%				\smallskip
%			\resumesection{Awards}{For an updated list, please visit \href{https://www.alextac.com/awards}{www.alextac.com/awards}}
	\award{2019}{SmallKat Major Qualifying Project (MQP) Honerable Mention (2nd best capstone project in Robotics Engineering Department)}
	\award{2018}{Dean’s List at WPI (Spring 2018)}
	\award{2018}{Inducted in Rho Beta Epsilon Robotics Engineering Honor Society}
	\award{2017}{1st Place at HealthHacksRI at the University of Rhode Island for Project Drogo}
	\award{2017}{NASA Space Robotics Challenge Team Finalist}
	\award{2016}{2nd Place at the Intel International Science and Engineering Fair (ISEF) in the category of Applied Mechanics}
	\award{2016}{Google International Science Fair Regional Finalist (top 100 in the world)}
	\award{2016}{International Council on Systems Engineering (INCOSE) First Award for “best interdisciplinary project that can produce technologically appropriate solution that meet societal needs” at the ISEF}
	\award{2016}{GE Fallonventions Award and participation on NBC’s Tonight Show starring Jimmy Fallon (aired on April 11, 2016). \href{https://alextac.com/personal/awards/major-awards/ge-fallonventions-on-the-tonight-show-starring-jimmy-fallon/}{See it here}}
	\award{2015 \& 2016}{Sweepstakes Award winner (1st place overall) and 1st place in Engineering at the Central California Science, Math, and Engineering Fair}
	\award{2015}{National Honor Society Inductee and California Scholarship Federation Member}
	\award{2015}{Institute of Electrical and Electronics Engineers President’s Scholarship Award at Intel Science and Engineering Fair for “an outstanding project demonstrating an understanding of electrical engineering, electronics engineering, and computer science.”}
	\award{2015}{1st place in the category of Applied Mechanics at the California State Science Fair}
%		\end{leftcolumn}
%	
%		\begin{rightcolumn} \raggedright
%			\cvsection{Qualifications Summary}
Robotics engineering student with significant experience in 3D CAD design and extensive knowledge of
multiple programing languages in multidisciplinary applications
%				\smallskip
%			\resumesection{Education}{}
\event{M.S. in Computer Science}{Stanford University}{Sept 1997 -- June 1999}{}
\divider

\event{B.S. in Symbolic Systems}{Stanford University}{Sept 1993 -- June 1997}{}
%				\smallskip
%			\resumesection{Technical Skills}{}
\skillsection{Robotics}{Software Development, Mechanical Design, Electrical Design, Agile Project Management (SCRUM)}

\skillsection{3D CAD}{Design and Simulation in Autodesk Inventor [7 years], Dassault SolidWorks [5 years], and PTC Creo Parametric (ProE) [1 year]}

\skillsection{Programming}{ROS (C++ \& Python), Android (Java), Pipelines (in Unix/ Linux) and Git}

\skillsection{Microprocessor \& Single-Board Computers}{Raspberry Pi, BeagleBone Black/Blue, NVIDIA TX1/2,ESP32 \& ESP8266, Device Trees in Linux}

\skillsection{Other Software Experience}{Linux (Debian/Ubuntu), MathCAD and MATLAB, Adobe Creative Suite(Photoshop \& Premiere), Excel Macros Programming}
\\
%				\smallskip
%			\resumesection{Leadership Experience}{}
	\leadership{2019-2020}{Officer board member for the Rho Beta Epsilon Robotics Honor Society}
	\newline
	\leadership{2016}{Leadership Practice at WPI: analyzed business and leadership practices for an on-campus organization with Prof. Sharon Wulf}
	\newline
	\leadership{2012-16}{ Lead Technical Director, Build and Pit Team Leader for FIRST FRC Robotics Team 2761}
%				\smallskip
%			\resumesection{Hobbies}{}
Robots, Coding, Tennis, Ultimate Frisbee, Skiing, Fishing, Camping
%		\end{rightcolumn}
%	\end{paracol}
\end{document}
